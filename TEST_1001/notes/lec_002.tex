\documentclass[12pt, letterpaper]{article}

%%%% GRAPHICS %%%%
\usepackage{tikz}
\usetikzlibrary{arrows.meta}
\usepackage{tikz-3dplot}
\usepackage{graphicx}
\usepackage{pgfplots}
  \pgfplotsset{compat=1.18}

%%%% FIGURES %%%%
\usepackage{subcaption}
\usepackage{wrapfig}
\usepackage{float}
\usepackage[skip=5pt, font=footnotesize]{caption}

%%%% FORMATTING %%%%
\usepackage{parskip}
\usepackage{tcolorbox}
\usepackage{ulem}

%%%% TABLE FORMATTING %%%%
\usepackage{tabularray}
\UseTblrLibrary{booktabs}

%%%% MATH AND LOGIC %%%%
\usepackage{xifthen}
\usepackage{amsmath}

%%%% TEXT AND SYMBOLS %%%%
\usepackage[T1]{fontenc}
\usepackage{textcomp}
\usepackage{gensymb}

%%%% OTHER %%%%
\usepackage{standalone}

%%%% STYLES %%%%

% Packages
\usepackage{fullpage}
\usepackage{titlesec}
\usepackage[rgb]{xcolor}
\selectcolormodel{natural}
\usepackage{ninecolors}
\selectcolormodel{rgb}
\usepackage{darkmode}
  \enabledarkmode
% \usepackage{minted}
% \IfDarkModeTF{\usemintedstyle{dracula}}{\usemintedstyle{igor}}

% Colors
\IfDarkModeTF{\definecolor{pg}{HTML}{283133}}{\definecolor{pg}{HTML}{FFFFFF}}
\IfDarkModeTF{\definecolor{fg}{HTML}{FFFFFF}}{\definecolor{fg}{HTML}{283133}}
\IfDarkModeTF{\definecolor{bg}{HTML}{283133}}{\definecolor{bg}{HTML}{FFFFFF}}
\IfDarkModeTF{\definecolor{re}{HTML}{D20F39}}{\definecolor{re}{HTML}{D20F39}}
\IfDarkModeTF{\definecolor{gr}{HTML}{40A02B}}{\definecolor{gr}{HTML}{40A02B}}
\IfDarkModeTF{\definecolor{ye}{HTML}{DF8E1D}}{\definecolor{ye}{HTML}{DF8E1D}}
\IfDarkModeTF{\definecolor{or}{HTML}{FE640B}}{\definecolor{or}{HTML}{FE640B}}
\IfDarkModeTF{\definecolor{bl}{HTML}{1E66F5}}{\definecolor{bl}{HTML}{1E66F5}}
\IfDarkModeTF{\definecolor{ma}{HTML}{8839EF}}{\definecolor{ma}{HTML}{8839EF}}
\IfDarkModeTF{\definecolor{cy}{HTML}{209FB5}}{\definecolor{cy}{HTML}{209FB5}}
\IfDarkModeTF{\definecolor{pi}{HTML}{F4B8E4}}{\definecolor{pi}{HTML}{EA76CB}}

\usepackage{nameref}
\makeatletter
\newcommand*{\currentname}{\@currentlabelname}
\makeatother

\titleformat{\section}
  {\normalfont\scshape\Large\bfseries}
  {\thesection}
  {0.75em}
  {}

\titleformat{\subsection}
  {\normalfont\scshape\large\bfseries}
  {\thesubsection}
  {0.75em}
  {}

\titleformat{\subsubsection}
  {\normalfont\scshape\normalsize\bfseries}
  {\thesubsubsection}
  {0.75em}
  {}

% Formula
\newcounter{formula}[section]
\newenvironment{formula}[1]{
  \stepcounter{formula}
  \begin{tcolorbox}[
    standard jigsaw, % Allows opacity
    colframe={\IfDarkModeTF{white}{black}},
    boxrule=1px,
    colback=black,
    opacityback=0,
    sharp corners,
    sidebyside,
    righthand width=18px,
    coltext={\IfDarkModeTF{white}{black}}
  ]
  \centering
  \textbf{\uline{#1}}
}{
  \tcblower
  \textbf{\thesection.\theformula}
  \end{tcolorbox}
}

% Definition
\newcounter{definition}[section]
\newenvironment{definition}[1]{
  \stepcounter{definition}
  \begin{tcolorbox}[
    standard jigsaw, % Allows opacity
    colframe={\IfDarkModeTF{white}{black}},
    boxrule=1px,
    colback=black,
    opacityback=0,
    sharp corners,
    coltext={\IfDarkModeTF{white}{black}}
  ]
  \textbf{#1 \hfill \thesection.\thedefinition}
  \vspace{5px}
  \hrule
  \vspace{5px}
  \noindent
}{
  \end{tcolorbox}
}

% Example Problem
\newcounter{example}[section]
\newenvironment{example}{
  \stepcounter{example}
  \begin{tcolorbox}[
    standard jigsaw, % Allows opacity
    colframe={\IfDarkModeTF{white}{black}},
    boxrule=1px,
    colback=black,
    opacityback=0,
    sharp corners,
    coltext={\IfDarkModeTF{white}{black}}
  ]
  \textbf{Example \hfill \thesection.\theexample}
  \vspace{5px}
  \hrule
  \vspace{5px}
  \noindent
}{
  \end{tcolorbox}
}

\tikzset{
  cubeBorder/.style=\IfDarkModeTF{white}{black},
  cubeFilling/.style={\IfDarkModeTF{white!20!black}{black!20!white}, opacity=0.25},
  gridLine/.style={very thin, gray},
  graphLine/.style={-latex, thick, \IfDarkModeTF{white}{black}},
}

\pgfplotsset{
  basicAxis/.style={
    grid,
    major grid style={line width=.2pt,draw=\IfDarkModeTF{white!50!black}{black!50!black}},
    axis lines = box,
    axis line style = {line width = 1px},
  }
}

%%%% REFERENCES %%%%
\usepackage{hyperref}
\hypersetup{
  colorlinks  = true,
  linkcolor   = pi,
  anchorcolor = pi,
  citecolor   = pi,
  filecolor   = pi,
  menucolor   = pi,
  runcolor    = pi,
  urlcolor    = pi,
}

\author{Ethan Anthony}


\title{Lecture 002}
\date{August 20, 2024}

\begin{document}

If velocity is a measurement of the rate of change of position, what is the measurement
of the rate of change of velocity?

\begin{definition}{Acceleration}
  Acceleration is a change in velocity over time. It is the measurement of the rate of
  change of velocity.
\end{definition}

It is important to remember that, just like velocity, acceleration is also a vector meaning
that is measures both a magnitude and a direction.

\begin{formula}{Average Acceleration}
  \begin{align*}
    \textup{Average acceleration} &= \textup{change in velocity over time elapsed} \\
    \textup{Average acceleration} &= \frac{\textup{change in velocity}}{\textup{time elapsed}} \\
    \overrightarrow{a}            &= \frac{\Delta v}{\Delta t}
  \end{align*}
\end{formula}

The direction of acceleration is the direction of the \textbf{change} of velocity. Notably,
the direction of velocity has no bearing on the direction of acceleration.

In Figure \ref{fig:007}, the direction of acceleration is the same as the direction of
velocity since the velocity is the velocity is going in the same direction from $t=1$ to
$t=2$ and it increases in magnitude.

In Figure \ref{fig:008}, the direction of acceleration is in the opposite direction of the
velocity of the object since the magnitude of velocity at $t=2$ is less than the magnitude
at $t=1$ but still in the same direction.

\begin{figure}[H]
  \centering
  \begin{subfigure}[H]{0.4\textwidth}
    \centering
    \includestandalone{figures/fig_007}
    \caption{Acceleration aligned with velocity}
    \label{fig:007}
  \end{subfigure}
  \begin{subfigure}[H]{0.4\textwidth}
    \centering
    \includestandalone{figures/fig_008}
    \caption{Acceleration opposite of velocity}
    \label{fig:008}
  \end{subfigure}
  \caption{Acceleration vs. Velocity}
  \label{fig:accVsVel}
\end{figure}

\subsection{Kinematics}

\begin{definition}{Kinematics}
  A branch of physics dedicated to describing the mechanics of motion.
\end{definition}

Assuming that we have constant acceleration, we can use \textbf{Kinematic Equations} to
describe motion.

\begin{formula}{Kinematic Equations}
  \begin{align*}
    v_{final} &= v_{initial}+a \cdot \Delta t                                                                            \\
    v         &= v_{0}+a \cdot \Delta t                                                                                  \\
  \end{align*}
  \vspace{-25pt}
  \hrule
  \begin{align*}
    x_{final} &= x_{initial}+v_{initial}\cdot (t_{final} - t_{initial})+\frac{1}{2}a \cdot (t_{final} - t_{initial})^{2} \\
    x         &= x_{0}+v_{0}\cdot \Delta t+\frac{1}{2}a \cdot \Delta t^{2}                                               \\
  \end{align*}
  \vspace{-25pt}
  \hrule
  \begin{align*}
    v_{final}^{2} &= v_{initial}^{2}+2a(x_{final}-x_{initial})                                                           \\
    v^{2}         &= v_{0}^{2}+2a(x-x_{0})                                                                               \\
  \end{align*}
  \vspace{-35pt}
\end{formula}

Kinematics equations are used extensively in predicting motion. Understanding these equations
is very important.

\end{document}
