\documentclass[12pt, letterpaper]{article}

%%%% GRAPHICS %%%%
\usepackage{tikz}
\usetikzlibrary{arrows.meta}
\usepackage{tikz-3dplot}
\usepackage{graphicx}
\usepackage{pgfplots}
  \pgfplotsset{compat=1.18}

%%%% FIGURES %%%%
\usepackage{subcaption}
\usepackage{wrapfig}
\usepackage{float}
\usepackage[skip=5pt, font=footnotesize]{caption}

%%%% FORMATTING %%%%
\usepackage{parskip}
\usepackage{tcolorbox}
\usepackage{ulem}

%%%% TABLE FORMATTING %%%%
\usepackage{tabularray}
\UseTblrLibrary{booktabs}

%%%% MATH AND LOGIC %%%%
\usepackage{xifthen}
\usepackage{amsmath}

%%%% TEXT AND SYMBOLS %%%%
\usepackage[T1]{fontenc}
\usepackage{textcomp}
\usepackage{gensymb}

%%%% OTHER %%%%
\usepackage{standalone}

%%%% STYLES %%%%

% Packages
\usepackage{fullpage}
\usepackage{titlesec}
\usepackage[rgb]{xcolor}
\selectcolormodel{natural}
\usepackage{ninecolors}
\selectcolormodel{rgb}
\usepackage{darkmode}
  \enabledarkmode
% \usepackage{minted}
% \IfDarkModeTF{\usemintedstyle{dracula}}{\usemintedstyle{igor}}

% Colors
\IfDarkModeTF{\definecolor{pg}{HTML}{283133}}{\definecolor{pg}{HTML}{FFFFFF}}
\IfDarkModeTF{\definecolor{fg}{HTML}{FFFFFF}}{\definecolor{fg}{HTML}{283133}}
\IfDarkModeTF{\definecolor{bg}{HTML}{283133}}{\definecolor{bg}{HTML}{FFFFFF}}
\IfDarkModeTF{\definecolor{re}{HTML}{D20F39}}{\definecolor{re}{HTML}{D20F39}}
\IfDarkModeTF{\definecolor{gr}{HTML}{40A02B}}{\definecolor{gr}{HTML}{40A02B}}
\IfDarkModeTF{\definecolor{ye}{HTML}{DF8E1D}}{\definecolor{ye}{HTML}{DF8E1D}}
\IfDarkModeTF{\definecolor{or}{HTML}{FE640B}}{\definecolor{or}{HTML}{FE640B}}
\IfDarkModeTF{\definecolor{bl}{HTML}{1E66F5}}{\definecolor{bl}{HTML}{1E66F5}}
\IfDarkModeTF{\definecolor{ma}{HTML}{8839EF}}{\definecolor{ma}{HTML}{8839EF}}
\IfDarkModeTF{\definecolor{cy}{HTML}{209FB5}}{\definecolor{cy}{HTML}{209FB5}}
\IfDarkModeTF{\definecolor{pi}{HTML}{F4B8E4}}{\definecolor{pi}{HTML}{EA76CB}}

\usepackage{nameref}
\makeatletter
\newcommand*{\currentname}{\@currentlabelname}
\makeatother

\titleformat{\section}
  {\normalfont\scshape\Large\bfseries}
  {\thesection}
  {0.75em}
  {}

\titleformat{\subsection}
  {\normalfont\scshape\large\bfseries}
  {\thesubsection}
  {0.75em}
  {}

\titleformat{\subsubsection}
  {\normalfont\scshape\normalsize\bfseries}
  {\thesubsubsection}
  {0.75em}
  {}

% Formula
\newcounter{formula}[section]
\newenvironment{formula}[1]{
  \stepcounter{formula}
  \begin{tcolorbox}[
    standard jigsaw, % Allows opacity
    colframe={\IfDarkModeTF{white}{black}},
    boxrule=1px,
    colback=black,
    opacityback=0,
    sharp corners,
    sidebyside,
    righthand width=18px,
    coltext={\IfDarkModeTF{white}{black}}
  ]
  \centering
  \textbf{\uline{#1}}
}{
  \tcblower
  \textbf{\thesection.\theformula}
  \end{tcolorbox}
}

% Definition
\newcounter{definition}[section]
\newenvironment{definition}[1]{
  \stepcounter{definition}
  \begin{tcolorbox}[
    standard jigsaw, % Allows opacity
    colframe={\IfDarkModeTF{white}{black}},
    boxrule=1px,
    colback=black,
    opacityback=0,
    sharp corners,
    coltext={\IfDarkModeTF{white}{black}}
  ]
  \textbf{#1 \hfill \thesection.\thedefinition}
  \vspace{5px}
  \hrule
  \vspace{5px}
  \noindent
}{
  \end{tcolorbox}
}

% Example Problem
\newcounter{example}[section]
\newenvironment{example}{
  \stepcounter{example}
  \begin{tcolorbox}[
    standard jigsaw, % Allows opacity
    colframe={\IfDarkModeTF{white}{black}},
    boxrule=1px,
    colback=black,
    opacityback=0,
    sharp corners,
    coltext={\IfDarkModeTF{white}{black}}
  ]
  \textbf{Example \hfill \thesection.\theexample}
  \vspace{5px}
  \hrule
  \vspace{5px}
  \noindent
}{
  \end{tcolorbox}
}

\tikzset{
  cubeBorder/.style=\IfDarkModeTF{white}{black},
  cubeFilling/.style={\IfDarkModeTF{white!20!black}{black!20!white}, opacity=0.25},
  gridLine/.style={very thin, gray},
  graphLine/.style={-latex, thick, \IfDarkModeTF{white}{black}},
}

\pgfplotsset{
  basicAxis/.style={
    grid,
    major grid style={line width=.2pt,draw=\IfDarkModeTF{white!50!black}{black!50!black}},
    axis lines = box,
    axis line style = {line width = 1px},
  }
}

%%%% REFERENCES %%%%
\usepackage{hyperref}
\hypersetup{
  colorlinks  = true,
  linkcolor   = pi,
  anchorcolor = pi,
  citecolor   = pi,
  filecolor   = pi,
  menucolor   = pi,
  runcolor    = pi,
  urlcolor    = pi,
}

\author{Ethan Anthony}


\title{Lecture 005}
\date{August 20, 2024}

\begin{document}

\subsection{Units and Conversion}

All measurements, in order to have any absolute significance, must be associated with a unit.
To standardize the way units are used, there is the \textbf{SI System of Units}, providing
a standard unit for every quantity.

The SI Base units are what every other unit is derived from. They are listed in full in
Figure \ref{tbl:siBaseUnits}.

\begin{figure}[H]
  \begin{center}
    \begin{tblr}{lcc}
      \toprule
      \textbf{Base Quantity}    & \textbf{Name} & \textbf{Symbol} \\
      \midrule
      Length                    & meter         & m               \\
      Mass                      & kilogram      & kg              \\
      Time                      & second        & s               \\
      Electric Current          & ampere        & A               \\
      Thermodynamic Temperature & kelvin        & K               \\
      Amount of Substance       & mole          & mol             \\
      Luminous Intensity        & candela       & cd              \\
      \bottomrule
    \end{tblr}
    \caption{SI Base Units}
    \label{tbl:siBaseUnits}
  \end{center}
\end{figure}

Using the SI Base Units, there is also a set of \textbf{SI Derived Units}, meaning that they are
standard units that are created through combining the SI Base Units in different ways. They
are listed in Figure \ref{tbl:siDerivedUnits}.

\begin{figure}[H]
  \begin{center}
    \begin{tblr}{lccc}
      \toprule
      \textbf{Derived Quantity} & \textbf{Name}  & \textbf{Symbol} & \textbf{Equivalent Base Units} \\
      \midrule
      Frequency                 & hertz          & Hz              & $s^{-1}$                       \\
      Force                     & newtown        & N               & $\frac{m \cdot kg}{s^2}$       \\
      Pressure                  & pascal         & Pa              & $\frac{N}{m^2}$                \\
      Energy                    & joule          & J               & $N \cdot m$                    \\
      Power                     & watt           & W               & $\frac{J}{s}$                  \\
      Electric Charge           & coulomb        & C               & $s \cdot A$                    \\
      Electric Potential        & volt           & V               & $\frac{W}{A}$                  \\
      Electric Resistance       & ohm            & $\Omega$        & $\frac{V}{A}$                  \\
      Celsius Temperature       & degree Celsius & $\celsius$      & $K$                            \\
      \bottomrule
    \end{tblr}
    \caption{SI Derived Units}
    \label{tbl:siDerivedUnits}
  \end{center}
\end{figure}

Lastly, there is a set of \textbf{SI Unit Prefixes}, used to modify the magnitude of a unit.
The range of prefixes is listed in Figure \ref{tbl:siPrefix}.

\begin{figure}[H]
  \begin{center}
    \begin{tblr}{lccl}
      \toprule
      \textbf{Factor} & \textbf{Name} & \textbf{Symbol} & \textbf{Numerical Value} \\
      \midrule
      $10^{12}$       & tera          & T               & $1\ 000\ 000\ 000\ 000$  \\
      $10^{9}$        & giga          & G               & $1\ 000\ 000\ 000$       \\
      $10^{6}$        & mega          & M               & $1\ 000\ 000$            \\
      $10^{3}$        & kilo          & k               & $1\ 000$                 \\
      $10^{2}$        & hecto         & h               & $100$                    \\
      $10^{1}$        & deka          & da              & $10$                     \\
      $10^{-1}$       & deci          & d               & $0.1$                    \\
      $10^{-2}$       & centi         & c               & $0.01$                   \\
      $10^{-3}$       & milli         & m               & $0.001$                  \\
      $10^{-6}$       & micro         & $\mu$           & $0.000\ 001$             \\
      $10^{-9}$       & nano          & n               & $0.000\ 000\ 001$        \\
      $10^{-12}$      & pico          & p               & $0.000\ 000\ 000\ 001$   \\
      \bottomrule
    \end{tblr}
    \caption{SI Unit Prefixes}
    \label{tbl:siPrefix}
  \end{center}
\end{figure}

\subsection{Vectors}

\begin{definition}{Vector}
  Any measurement that has both a \textbf{size} and a \textbf{direction}, such as velocity,
  acceleration, displacement, position, force, momentum, etc.
\end{definition}

\begin{figure}[H]
  \centering
  \includestandalone{figures/fig_013}
\caption{Components of Vectors}
\label{fig:013}
\end{figure}

In Figure \ref{fig:013}, an object moves from $p_1$ to $p_2$, changing both its $x$ and $y$
positions. It's vector displacement is northeast, but can also be represented as two
component vectors, one moving east along the $x$-axis, and one moving north along the $y$-axis.
\\
Using trigonometry, the size of the components can be determined:
\begin{formula}{Vector Components}
  \begin{tcolorbox}[
    standard jigsaw, % Allows opacity
    colframe=fg,
    boxrule=0px,
    opacityback=0,
    sidebyside,
    lefthand width=0.43\textwidth,
    coltext=fg,
  ]
  \begin{center}
    \textbf{Vertical Component}
  \end{center}
  \begin{align*}
    sin(\theta)         &= \frac{\textup{opposite}}{\textup{hypotenuse}} \\
    sin(\theta)         &= \frac{D_y}{D} \\
    D_y &= sin(\theta) \cdot D \\
  \end{align*}
  \tcblower
  \begin{center}
    \textbf{Horizontal Component}
  \end{center}
  \begin{align*}
    cos(\theta)         &= \frac{\textup{adjacent}}{\textup{hypotenuse}} \\
    cos(\theta)         &= \frac{D_x}{D} \\
    D_x &= cos(\theta) \cdot D \\
  \end{align*}
  \end{tcolorbox}
\end{formula}

Multiple vectors can be added up. In essence, this is done by adding all like components
with like components, then combining what components you have left.

\begin{figure}[H]
  \centering
  \includestandalone{figures/fig_014}
\caption{Combining Multiple Vectors}
\label{fig:014}
\end{figure}

Just as you can add vectors, so too can you multiply them. Multiplying a vector by a constant
scales the magnitude of the vector by that constant without affecting the direction of it.

Lastly, notation. Notation for a vector requires that an $\overrightarrow{\textup{arrow}}$ is put over
any vector variable to show that it's a vector. If instead of an arrow, there's a $\widehat{\textup{h}}$at on
the vector, that means it's a unit vector.

\begin{definition}{Unit Vector}
  A unit vector is one with a magnitude of 1, used to signify a direction without really
  affecting the magnitude. They are usually represented by $\hat{x}$, $\hat{y}$, and $\hat{z}$ to point in their
  respective directions. Or by $\hat{i}$, $\hat{j}$, and $\hat{k}$ instead.
\end{definition}

The example in Figure \ref{fig:013} can be written out in terms of its components using unit
vectors.

\begin{equation*}
  \overrightarrow{D}=D_x\hat{i}+D_y\hat{j}
\end{equation*}

\end{document}
