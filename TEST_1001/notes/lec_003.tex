\documentclass[12pt]{article}

%%%% GRAPHICS %%%%
\usepackage{tikz}
\usetikzlibrary{arrows.meta}
\usepackage{tikz-3dplot}
\usepackage{graphicx}
\usepackage{pgfplots}
  \pgfplotsset{compat=1.18}

%%%% FIGURES %%%%
\usepackage{subcaption}
\usepackage{wrapfig}
\usepackage{float}
\usepackage[skip=5pt, font=footnotesize]{caption}

%%%% FORMATTING %%%%
\usepackage{parskip}
\usepackage{tcolorbox}
\usepackage{ulem}

%%%% TABLE FORMATTING %%%%
\usepackage{tabularray}
\UseTblrLibrary{booktabs}

%%%% MATH AND LOGIC %%%%
\usepackage{xifthen}
\usepackage{amsmath}

%%%% TEXT AND SYMBOLS %%%%
\usepackage[T1]{fontenc}
\usepackage{textcomp}
\usepackage{gensymb}

%%%% OTHER %%%%
\usepackage{standalone}

%%%% STYLES %%%%

% Packages
\usepackage{fullpage}
\usepackage{titlesec}
\usepackage[rgb]{xcolor}
\selectcolormodel{natural}
\usepackage{ninecolors}
\selectcolormodel{rgb}
\usepackage{darkmode}
  \enabledarkmode
% \usepackage{minted}
% \IfDarkModeTF{\usemintedstyle{dracula}}{\usemintedstyle{igor}}

% Colors
\IfDarkModeTF{\definecolor{pg}{HTML}{283133}}{\definecolor{pg}{HTML}{FFFFFF}}
\IfDarkModeTF{\definecolor{fg}{HTML}{FFFFFF}}{\definecolor{fg}{HTML}{283133}}
\IfDarkModeTF{\definecolor{bg}{HTML}{283133}}{\definecolor{bg}{HTML}{FFFFFF}}
\IfDarkModeTF{\definecolor{re}{HTML}{D20F39}}{\definecolor{re}{HTML}{D20F39}}
\IfDarkModeTF{\definecolor{gr}{HTML}{40A02B}}{\definecolor{gr}{HTML}{40A02B}}
\IfDarkModeTF{\definecolor{ye}{HTML}{DF8E1D}}{\definecolor{ye}{HTML}{DF8E1D}}
\IfDarkModeTF{\definecolor{or}{HTML}{FE640B}}{\definecolor{or}{HTML}{FE640B}}
\IfDarkModeTF{\definecolor{bl}{HTML}{1E66F5}}{\definecolor{bl}{HTML}{1E66F5}}
\IfDarkModeTF{\definecolor{ma}{HTML}{8839EF}}{\definecolor{ma}{HTML}{8839EF}}
\IfDarkModeTF{\definecolor{cy}{HTML}{209FB5}}{\definecolor{cy}{HTML}{209FB5}}
\IfDarkModeTF{\definecolor{pi}{HTML}{F4B8E4}}{\definecolor{pi}{HTML}{EA76CB}}

\usepackage{nameref}
\makeatletter
\newcommand*{\currentname}{\@currentlabelname}
\makeatother

\titleformat{\section}
  {\normalfont\scshape\Large\bfseries}
  {\thesection}
  {0.75em}
  {}

\titleformat{\subsection}
  {\normalfont\scshape\large\bfseries}
  {\thesubsection}
  {0.75em}
  {}

\titleformat{\subsubsection}
  {\normalfont\scshape\normalsize\bfseries}
  {\thesubsubsection}
  {0.75em}
  {}

% Formula
\newcounter{formula}[section]
\newenvironment{formula}[1]{
  \stepcounter{formula}
  \begin{tcolorbox}[
    standard jigsaw, % Allows opacity
    colframe={\IfDarkModeTF{white}{black}},
    boxrule=1px,
    colback=black,
    opacityback=0,
    sharp corners,
    sidebyside,
    righthand width=18px,
    coltext={\IfDarkModeTF{white}{black}}
  ]
  \centering
  \textbf{\uline{#1}}
}{
  \tcblower
  \textbf{\thesection.\theformula}
  \end{tcolorbox}
}

% Definition
\newcounter{definition}[section]
\newenvironment{definition}[1]{
  \stepcounter{definition}
  \begin{tcolorbox}[
    standard jigsaw, % Allows opacity
    colframe={\IfDarkModeTF{white}{black}},
    boxrule=1px,
    colback=black,
    opacityback=0,
    sharp corners,
    coltext={\IfDarkModeTF{white}{black}}
  ]
  \textbf{#1 \hfill \thesection.\thedefinition}
  \vspace{5px}
  \hrule
  \vspace{5px}
  \noindent
}{
  \end{tcolorbox}
}

% Example Problem
\newcounter{example}[section]
\newenvironment{example}{
  \stepcounter{example}
  \begin{tcolorbox}[
    standard jigsaw, % Allows opacity
    colframe={\IfDarkModeTF{white}{black}},
    boxrule=1px,
    colback=black,
    opacityback=0,
    sharp corners,
    coltext={\IfDarkModeTF{white}{black}}
  ]
  \textbf{Example \hfill \thesection.\theexample}
  \vspace{5px}
  \hrule
  \vspace{5px}
  \noindent
}{
  \end{tcolorbox}
}

\tikzset{
  cubeBorder/.style=\IfDarkModeTF{white}{black},
  cubeFilling/.style={\IfDarkModeTF{white!20!black}{black!20!white}, opacity=0.25},
  gridLine/.style={very thin, gray},
  graphLine/.style={-latex, thick, \IfDarkModeTF{white}{black}},
}

\pgfplotsset{
  basicAxis/.style={
    grid,
    major grid style={line width=.2pt,draw=\IfDarkModeTF{white!50!black}{black!50!black}},
    axis lines = box,
    axis line style = {line width = 1px},
  }
}

%%%% REFERENCES %%%%
\usepackage{hyperref}
\hypersetup{
  colorlinks  = true,
  linkcolor   = pi,
  anchorcolor = pi,
  citecolor   = pi,
  filecolor   = pi,
  menucolor   = pi,
  runcolor    = pi,
  urlcolor    = pi,
}

\author{Ethan Anthony}


\title{Lecture 003}
\date{August 20, 2024}

\begin{document}

\subsection{Examples}

\begin{example}
  A car starts at rest, and then accelerates uniformly. After 5 seconds, it passes the 25
  mater mark on the track. What is the car's acceleration?
  \begin{tcolorbox}[
    standard jigsaw, % Allows opacity
    colframe=fg,
    boxrule=0px,
    opacityback=0,
    sidebyside,
    lefthand width=75px,
    coltext=fg,
  ]
  \textbf{Known Values}
  \begin{align*}
    x_i &= 0m             \\
    x_f &= 25m            \\
    v_i &= 0 \frac{m}{s}  \\
    t_i &= 0s             \\
    t_f &= 5s             \\
  \end{align*}
  \tcblower
  \begin{align*}
    x_f &= x_{i}+v_{i}t+\frac{1}{2}at^{2} \\
    25  &= 0+0(5-0)+\frac{1}{2}a(5-0)^{2} \\
    25  &= 0+0+\frac{25}{2}a \\
    \frac{1}{a}  &= \frac{25}{2 \cdot 25}
    \\[5pt]
    a  &= \frac{2 \cdot 25}{25}
    \\[5pt]
    a  &= 2 \frac{m}{s^2} \\
  \end{align*}
  \end{tcolorbox}
\end{example}

\begin{example}
  The driver then releases the gas and slams on the brakes. She stops after traveling an
  additional 15 $m$. What was the acceleration during the braking period?
  \begin{tcolorbox}[
    standard jigsaw, % Allows opacity
    colframe=fg,
    boxrule=0px,
    opacityback=0,
    sidebyside,
    lefthand width=75px,
    coltext=fg,
  ]
  \textbf{Known Values}
  \begin{align*}
    x_i &= 0m                \\
    x_f &= 25m               \\
    v_i &= 0 \frac{m}{s}     \\
    a   &= 2 \frac{m}{s^2}   \\
    t_i &= 0s                \\
    t_f &= 5s
  \end{align*}
  \tcblower
  At the time the brakes are engaged, what is the velocity?
  \begin{align*}
    v_f &= v_i+at         \\
    v_f &= 0+2 \cdot 5    \\
    v_f &= 10 \frac{m}{s}
  \end{align*}
  What is the acceleration if she travels 15 $m$ to go from $v=10$ to $v=0$?
  \begin{align*}
    v_f^2 &= v_i^2 + 2a(x_f-x_i)        \\
    0^2   &= 10^2 + 2a(15)              \\
    a     &= \frac{-(10^2)}{2 \cdot 15} \\
    a     &= \frac{-100}{30}            \\
    a     &= -3.33 \frac{m}{s^2}
  \end{align*}
  \end{tcolorbox}
\end{example}

\begin{example}
  Earthquakes produce several types of shock waves. The most well-known are the P-waves 
  (primary, or pressure), and the S-waves (secondary, or shear). In the Earth’s crust, 
  P-waves travel at around $6.5 \frac{km}{s}$, while the S-waves move at about $3.5 \frac{km}{s}$. The time
  delay between the arrival of these two waves at a seismic recording station tells 
  geologists how far away the earthquake occurred. If this time delay is $33 s$, how far
  from the seismic station did the Earthquake occur?

  \begin{tcolorbox}[
    standard jigsaw, % Allows opacity
    colframe=fg,
    boxrule=0px,
    opacityback=0,
    sidebyside,
    lefthand width=75px,
    coltext=fg,
  ]
  \textbf{Known Values}
  \begin{align*}
    v_{\textup{p}}     &= 6.5 \frac{km}{s} \\
    v_{\textup{s}}     &= 3.5 \frac{km}{s} \\
    t_{\textup{delay}} &= 33 s
  \end{align*}
  \tcblower
  \begin{align*}
    x_f       &= x_i + v_it + \frac{1}{2}at^2 \\
    x_f - x_i &= v_it + \frac{1}{2}at^2 \\ 
    x         &= v_it
  \end{align*}
  How can we find $v_i$? Keeping in line with $t$ measuring the difference between the time for each wave, so can $v_i$ measure the difference between the velocities of both waves.
  \begin{align*}
    x &= (v_p-v_s)t  \\
    x &= (6.5-3.5)33 \\
    x &= 99km
  \end{align*}
  \end{tcolorbox}
\end{example}

\end{document}
