\documentclass[12pt, letterpaper]{article}

%%%% GRAPHICS %%%%
\usepackage{tikz}
\usetikzlibrary{arrows.meta}
\usepackage{tikz-3dplot}
\usepackage{graphicx}
\usepackage{pgfplots}
  \pgfplotsset{compat=1.18}

%%%% FIGURES %%%%
\usepackage{subcaption}
\usepackage{wrapfig}
\usepackage{float}
\usepackage[skip=5pt, font=footnotesize]{caption}

%%%% FORMATTING %%%%
\usepackage{parskip}
\usepackage{tcolorbox}
\usepackage{ulem}

%%%% TABLE FORMATTING %%%%
\usepackage{tabularray}
\UseTblrLibrary{booktabs}

%%%% MATH AND LOGIC %%%%
\usepackage{xifthen}
\usepackage{amsmath}

%%%% TEXT AND SYMBOLS %%%%
\usepackage[T1]{fontenc}
\usepackage{textcomp}
\usepackage{gensymb}

%%%% OTHER %%%%
\usepackage{standalone}

%%%% STYLES %%%%

% Packages
\usepackage{fullpage}
\usepackage{titlesec}
\usepackage[rgb]{xcolor}
\selectcolormodel{natural}
\usepackage{ninecolors}
\selectcolormodel{rgb}
\usepackage{darkmode}
  \enabledarkmode
% \usepackage{minted}
% \IfDarkModeTF{\usemintedstyle{dracula}}{\usemintedstyle{igor}}

% Colors
\IfDarkModeTF{\definecolor{pg}{HTML}{283133}}{\definecolor{pg}{HTML}{FFFFFF}}
\IfDarkModeTF{\definecolor{fg}{HTML}{FFFFFF}}{\definecolor{fg}{HTML}{283133}}
\IfDarkModeTF{\definecolor{bg}{HTML}{283133}}{\definecolor{bg}{HTML}{FFFFFF}}
\IfDarkModeTF{\definecolor{re}{HTML}{D20F39}}{\definecolor{re}{HTML}{D20F39}}
\IfDarkModeTF{\definecolor{gr}{HTML}{40A02B}}{\definecolor{gr}{HTML}{40A02B}}
\IfDarkModeTF{\definecolor{ye}{HTML}{DF8E1D}}{\definecolor{ye}{HTML}{DF8E1D}}
\IfDarkModeTF{\definecolor{or}{HTML}{FE640B}}{\definecolor{or}{HTML}{FE640B}}
\IfDarkModeTF{\definecolor{bl}{HTML}{1E66F5}}{\definecolor{bl}{HTML}{1E66F5}}
\IfDarkModeTF{\definecolor{ma}{HTML}{8839EF}}{\definecolor{ma}{HTML}{8839EF}}
\IfDarkModeTF{\definecolor{cy}{HTML}{209FB5}}{\definecolor{cy}{HTML}{209FB5}}
\IfDarkModeTF{\definecolor{pi}{HTML}{F4B8E4}}{\definecolor{pi}{HTML}{EA76CB}}

\usepackage{nameref}
\makeatletter
\newcommand*{\currentname}{\@currentlabelname}
\makeatother

\titleformat{\section}
  {\normalfont\scshape\Large\bfseries}
  {\thesection}
  {0.75em}
  {}

\titleformat{\subsection}
  {\normalfont\scshape\large\bfseries}
  {\thesubsection}
  {0.75em}
  {}

\titleformat{\subsubsection}
  {\normalfont\scshape\normalsize\bfseries}
  {\thesubsubsection}
  {0.75em}
  {}

% Formula
\newcounter{formula}[section]
\newenvironment{formula}[1]{
  \stepcounter{formula}
  \begin{tcolorbox}[
    standard jigsaw, % Allows opacity
    colframe={\IfDarkModeTF{white}{black}},
    boxrule=1px,
    colback=black,
    opacityback=0,
    sharp corners,
    sidebyside,
    righthand width=18px,
    coltext={\IfDarkModeTF{white}{black}}
  ]
  \centering
  \textbf{\uline{#1}}
}{
  \tcblower
  \textbf{\thesection.\theformula}
  \end{tcolorbox}
}

% Definition
\newcounter{definition}[section]
\newenvironment{definition}[1]{
  \stepcounter{definition}
  \begin{tcolorbox}[
    standard jigsaw, % Allows opacity
    colframe={\IfDarkModeTF{white}{black}},
    boxrule=1px,
    colback=black,
    opacityback=0,
    sharp corners,
    coltext={\IfDarkModeTF{white}{black}}
  ]
  \textbf{#1 \hfill \thesection.\thedefinition}
  \vspace{5px}
  \hrule
  \vspace{5px}
  \noindent
}{
  \end{tcolorbox}
}

% Example Problem
\newcounter{example}[section]
\newenvironment{example}{
  \stepcounter{example}
  \begin{tcolorbox}[
    standard jigsaw, % Allows opacity
    colframe={\IfDarkModeTF{white}{black}},
    boxrule=1px,
    colback=black,
    opacityback=0,
    sharp corners,
    coltext={\IfDarkModeTF{white}{black}}
  ]
  \textbf{Example \hfill \thesection.\theexample}
  \vspace{5px}
  \hrule
  \vspace{5px}
  \noindent
}{
  \end{tcolorbox}
}

\tikzset{
  cubeBorder/.style=\IfDarkModeTF{white}{black},
  cubeFilling/.style={\IfDarkModeTF{white!20!black}{black!20!white}, opacity=0.25},
  gridLine/.style={very thin, gray},
  graphLine/.style={-latex, thick, \IfDarkModeTF{white}{black}},
}

\pgfplotsset{
  basicAxis/.style={
    grid,
    major grid style={line width=.2pt,draw=\IfDarkModeTF{white!50!black}{black!50!black}},
    axis lines = box,
    axis line style = {line width = 1px},
  }
}

%%%% REFERENCES %%%%
\usepackage{hyperref}
\hypersetup{
  colorlinks  = true,
  linkcolor   = pi,
  anchorcolor = pi,
  citecolor   = pi,
  filecolor   = pi,
  menucolor   = pi,
  runcolor    = pi,
  urlcolor    = pi,
}

\author{Ethan Anthony}


\title{Lecture 004}
\date{August 20, 2024}

\begin{document}

\newpage
\section{Gravity}
\label{sec:gravity}
\subsection{Introduction to Gravity}
\label{ssec:introductionToGravity}

All freely falling objects have the same downward acceleration. Freely falling means that
gravity is the only force acting upon the object.

\begin{formula}{Gravity on Earth}
  \begin{equation*}
    g=9.8 \frac{m}{s^2}
  \end{equation*}
\end{formula}

\section{Fundamentals for Physics}
\label{sec:fundamentalsForPhysics}

\subsection{Calculus}
\label{ssec:calculus}

\begin{wrapfigure}[6]{r}{0.5\textwidth}
  \centering
  \includestandalone{figures/fig_009}
\caption{Derivative of a function}
\label{fig:009}
\end{wrapfigure}

Calculus is used to measure rates of change of continuous processes over a continuous range of
time. Specific to kinematics, it is used to model motion of objects through time.

\begin{definition}{Derivative}
  A derivative of a function is the \textbf{rate of change} of that function.
  \begin{gather*}
    \textup{Derivative of}\ f(x) = f'(x) = \frac{df}{dx} \\
    \frac{df}{dx} = \lim_{\Delta x \to 0} \frac{\Delta f}{\Delta x}
  \end{gather*}
\end{definition}

In Figure \ref{fig:009}, there is a function with two lines running tangent to it at different
points. These lines are the \textbf{derivative} of the function at the points they intersect, thus
showing the instantaneous slope of the function. If this function represents an object's
position over time, then the tangent lines would show the instantaneous velocity at a point
in time.

As it relates to physics, position ($x$) and velocity ($v$) are related by $v=\frac{dx}{dt}$.
In other words, $x'(t) = v(t)$.

\begin{figure}[H]
  \centering
  \begin{subfigure}[H]{0.3\textwidth}
    \centering
    \includestandalone{figures/fig_010}
    \caption{Position}
    \label{fig:010}
  \end{subfigure}
  \begin{subfigure}[H]{0.3\textwidth}
    \centering
    \includestandalone{figures/fig_011}
    \caption{Velocity}
    \label{fig:011}
  \end{subfigure}
  \begin{subfigure}[H]{0.3\textwidth}
    \centering
    \includestandalone{figures/fig_012}
    \caption{Acceleration}
    \label{fig:012}
  \end{subfigure}
  \caption{Position, velocity, and acceleration of an object accelerating constantly}
  \label{fig:posVelAcc}
\end{figure}

\end{document}
