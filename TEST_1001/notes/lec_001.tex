\documentclass[12pt]{article}

%%%% GRAPHICS %%%%
\usepackage{tikz}
\usetikzlibrary{arrows.meta}
\usepackage{tikz-3dplot}
\usepackage{graphicx}
\usepackage{pgfplots}
  \pgfplotsset{compat=1.18}

%%%% FIGURES %%%%
\usepackage{subcaption}
\usepackage{wrapfig}
\usepackage{float}
\usepackage[skip=5pt, font=footnotesize]{caption}

%%%% FORMATTING %%%%
\usepackage{tcolorbox}
\usepackage{ulem}

%%%% TABLE FORMATTING %%%%
\usepackage{tabularray}
\UseTblrLibrary{booktabs}

%%%% MATH AND LOGIC %%%%
\usepackage{xifthen}
\usepackage{amsmath}

%%%% TEXT AND SYMBOLS %%%%
\usepackage[T1]{fontenc}
\usepackage{textcomp}
\usepackage{gensymb}

%%%% OTHER %%%%
\usepackage{standalone}

%%%% STYLES %%%%

% Packages
\usepackage{fullpage}
\usepackage{titlesec}
\usepackage[rgb]{xcolor}
\selectcolormodel{natural}
\usepackage{ninecolors}
\selectcolormodel{rgb}
\usepackage{darkmode}
  \enabledarkmode
% \usepackage{minted}
% \IfDarkModeTF{\usemintedstyle{dracula}}{\usemintedstyle{igor}}

% Colors
\IfDarkModeTF{\definecolor{pg}{HTML}{283133}}{\definecolor{pg}{HTML}{FFFFFF}}
\IfDarkModeTF{\definecolor{fg}{HTML}{FFFFFF}}{\definecolor{fg}{HTML}{000000}}
\IfDarkModeTF{\definecolor{bg}{HTML}{000000}}{\definecolor{bg}{HTML}{FFFFFF}}
\IfDarkModeTF{\definecolor{re}{HTML}{D20F39}}{\definecolor{re}{HTML}{D20F39}}
\IfDarkModeTF{\definecolor{gr}{HTML}{40A02B}}{\definecolor{gr}{HTML}{40A02B}}
\IfDarkModeTF{\definecolor{ye}{HTML}{DF8E1D}}{\definecolor{ye}{HTML}{DF8E1D}}
\IfDarkModeTF{\definecolor{or}{HTML}{FE640B}}{\definecolor{or}{HTML}{FE640B}}
\IfDarkModeTF{\definecolor{bl}{HTML}{1E66F5}}{\definecolor{bl}{HTML}{1E66F5}}
\IfDarkModeTF{\definecolor{ma}{HTML}{8839EF}}{\definecolor{ma}{HTML}{8839EF}}
\IfDarkModeTF{\definecolor{cy}{HTML}{209FB5}}{\definecolor{cy}{HTML}{209FB5}}
\IfDarkModeTF{\definecolor{pi}{HTML}{EA76CB}}{\definecolor{pi}{HTML}{EA76CB}}

\usepackage{nameref}
\makeatletter
\newcommand*{\currentname}{\@currentlabelname}
\makeatother

\titleformat{\section}
  {\normalfont\scshape\Large\bfseries}
  {\thesection}
  {0.75em}
  {}

\titleformat{\subsection}
  {\normalfont\scshape\large\bfseries}
  {\thesubsection}
  {0.75em}
  {}

\titleformat{\subsubsection}
  {\normalfont\scshape\normalsize\bfseries}
  {\thesubsubsection}
  {0.75em}
  {}

% Formula
\newcounter{formula}[section]
\newenvironment{formula}[1]{
  \stepcounter{formula}
  \begin{tcolorbox}[
    standard jigsaw, % Allows opacity
    colframe={\IfDarkModeTF{white}{black}},
    boxrule=1px,
    colback=black,
    opacityback=0,
    sharp corners,
    sidebyside,
    righthand width=18px,
    coltext={\IfDarkModeTF{white}{black}}
  ]
  \centering
  \textbf{\uline{#1}}
}{
  \tcblower
  \textbf{\thesection.\theformula}
  \end{tcolorbox}
}

% Definition
\newcounter{definition}[section]
\newenvironment{definition}[1]{
  \stepcounter{definition}
  \begin{tcolorbox}[
    standard jigsaw, % Allows opacity
    colframe={\IfDarkModeTF{white}{black}},
    boxrule=1px,
    colback=black,
    opacityback=0,
    sharp corners,
    coltext={\IfDarkModeTF{white}{black}}
  ]
  \textbf{#1 \hfill \thesection.\thedefinition}
  \vspace{5px}
  \hrule
  \vspace{5px}
  \noindent
}{
  \end{tcolorbox}
}

% Example Problem
\newcounter{example}[section]
\newenvironment{example}{
  \stepcounter{example}
  \begin{tcolorbox}[
    standard jigsaw, % Allows opacity
    colframe={\IfDarkModeTF{white}{black}},
    boxrule=1px,
    colback=black,
    opacityback=0,
    sharp corners,
    coltext={\IfDarkModeTF{white}{black}}
  ]
  \textbf{Example \hfill \thesection.\theexample}
  \vspace{5px}
  \hrule
  \vspace{5px}
  \noindent
}{
  \end{tcolorbox}
}

\tikzset{
  cubeBorder/.style=\IfDarkModeTF{white}{black},
  cubeFilling/.style={\IfDarkModeTF{white!20!black}{black!20!white}, opacity=0.25},
  gridLine/.style={very thin, gray},
  graphLine/.style={-latex, thick, \IfDarkModeTF{white}{black}},
}

\pgfplotsset{
  basicAxis/.style={
    grid,
    major grid style={line width=.2pt,draw=\IfDarkModeTF{white!50!black}{black!50!black}},
    axis lines = box,
    axis line style = {line width = 1px},
  }
}

%%%% REFERENCES %%%%
\usepackage{hyperref}

\author{Ethan Anthony}


\title{Lecture 001}
\date{August 20, 2024}

\begin{document}

\section{Motion in One Dimension}

\subsection{Position}

If an object is in motion, what quantities must be known to predict its position at a
certain time? First, what is position?

\begin{definition}{Position}
  A description of where an object is located.
\end{definition}

In Figure \ref{fig:001}, there is an object at position $x=2.5$. This is defined based on
the number line that the object is positioned on. In Figure \ref{fig:002}, the object is
positioned at $x=5.5$ despite not having moved. This highlights the importance of recognizing
the reference frame when describing an object and its properties.

\begin{figure}[H]
  \centering
  \begin{subfigure}[H]{0.4\textwidth}
    \centering
    \includestandalone{figures/fig_001}
    \caption{Position is at $x=2.5$}
    \label{fig:001}
  \end{subfigure}
  \begin{subfigure}[H]{0.4\textwidth}
    \centering
    \includestandalone{figures/fig_002}
    \caption{Position is at $x=5.5$}
    \label{fig:002}
  \end{subfigure}
  \caption{Different Reference Frames}
  \label{fig:refFrame}
\end{figure}

\subsection{Motion}

\begin{definition}{Reference Frame}
  A reference frame is the context in which the values are described. This extends to the
  context of the motion, position, energy, etc. of the description.
\end{definition}

Objects not only have position, but also experience motion. When an object undergoes motion,
it is displaced. In Figure \ref{fig:003}, an object moved to the right (within our reference
frame) from $x=4$ to $x=7$, resulting in a displacement of $\Delta x=3$. In Figure
\ref{fig:004}, an object first moves from $x=5$ to $x=3$, then from $x=3$ to $x=7$, resulting
in a final displacement of $\Delta x=2$.

\begin{definition}{Displacement}
  An object's net \textbf{change in position}.
\end{definition}

\begin{figure}[H]
  \centering
  \begin{subfigure}[t]{0.4\textwidth}
    \centering
    \includestandalone{figures/fig_003}
    \caption{Motion to the right}
    \label{fig:003}
  \end{subfigure}
  \begin{subfigure}[t]{0.4\textwidth}
    \centering
    \includestandalone{figures/fig_004}
    \caption{Motion in multiple directions}
    \label{fig:004}
  \end{subfigure}
  \caption{Subsonic and sonic speeds}
  \label{fig:subandsonic}
\end{figure}

\begin{wrapfigure}[10]{l}{0.4\textwidth}
  \centering
  \includestandalone{figures/fig_005}
\caption{Displacement displayed graphically}
\label{fig:005}
\end{wrapfigure}

\newpage

In Figure \ref{fig:005}, the position ($x$) of two objects is graphed over time ($t$).
For object 1, its displacement is constantly changing throughout the entire time. However,
for object 2, its displacement starts by increasing, then at $t=c$, its displacement starts
decreasing until it ends with a negative displacement. For object 2, it maintains a
constant speed throughout the entire time, but its velocity changes. This is because speed
is a \textbf{scalar} measurement, while velocity is a \textbf{vector} meaning that speed
measures \textbf{magnitude} while velocity measures both \textbf{magnitude and direction}.

\vspace{12pt}
\hrule
\vspace{12pt}

How can speed and velocity be quantitatively represented? Through equations!

\begin{formula}{Speed and Velocity}
  \begin{align*}
    \textup{\textbf{Speed}}  & \textup{\ is distance traveled over time elapsed.}         \\
    \textup{Speed}           &= \frac{\textup{distance traveled}}{\textup{time elapsed}}  \\
    v                        &= \frac{x}{t}
  \end{align*}
  \begin{align*}
  \textup{\textbf{Velocity}} & \textup{\ is change in position over time elapsed}         \\
    \textup{Velocity}        &= \frac{\textup{change in position}}{\textup{time elapsed}} \\
    \overrightarrow{v}       &= \frac{\Delta \overrightarrow{x}}{\Delta t}
  \end{align*}
\end{formula}


\newpage

\subsection{Acceleration}

\begin{wrapfigure}{l}{0.45\textwidth}
  \centering
  \includestandalone{figures/fig_006}
  \caption{Instantaneous vs. Average Velocity}
  \label{fig:006}
\end{wrapfigure}

There is also a difference between an instantaneous value and an average value. In Figure
\ref{fig:006}, the average velocity of the object is represented by the dotted line,
showing a slow and negative average velocity. However, from the start to point $A$, the
velocity is fast and positive; from point $A$ to point $B$, the velocity is fast and
negative; and from point $B$ to point $C$, the velocity is slow and positive.

\begin{definition}{Instantaneous vs. Average}
  Instantaneous measures a value at a \textbf{single point} in time the same way calculus
  is used to find the slope of a function at a single point. Average measures the mean
  value over a period of time.
\end{definition}

\end{document}
